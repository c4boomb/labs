\input{../common/head.tex}
\def\passYear{2016}
\def\faculty{физико-технический институт}
\def\department{Кафедра информационной безопасности}
\def\departmentHead{Н. В. Грайворонский}
\def\kind{Дипломна робота}
\def\level{магістр}
\def\specialityCode{8.04030101}
\def\specialityTitle{Прикладная математика}
\def\theme{Сценарии обработки многоспектральных спутниковых изображений}
\def\gender{female}
\def\mentorGender{male}
\def\course{2}
\def\group{ФИ-41}
\def\name{Лавягина Ольга Алексеевна}
\def\mentorRank{}
\def\mentorName{Колотий Андрей Всеволодович}
\def\reviewerRank{Rank}
\def\reviewerName{Name}
\def\subject{Специальные разделы программирования}



\begin{document}

\import{1_title/}{title.tex}

\clearpage

\pagenumbering{gobble}
%\import{3_abstract/}{main.tex}

%\pagestyle{empty}
%\thispagestyle{empty}
%\tableofcontents

\clearpage
\pagenumbering{arabic}
\pagestyle{fancy}
\setcounter{page}{2}

\clearpage

\chapter{Задание}
\begin{itemize}
\item Зарегистрироваться на сайте GitHub, создать репозиторий, добавить в репозиторий код и данные из лабораторной работы №1, продемонстрировать навыки работы с системой контроля версий git на работе с проэктом GitHub;
\item создать веб-приложение с использованием модуля Spyre, которое позволит:
\begin{itemize}
\item выбрать часовой ряд VCI, TCI, VHI для наборы данных из лабораторной работы 1 (выпадающий список);
\item выбрать область, для которой будет выполняться анализ (выпадающий список);
\item указать интервал недель, за которые отбираются данные;
\item создать несколько вкладок для отображения таблицы с данными на графике хода индексов;
\end{itemize}
\item код разработанного приложения добавить к созданному репозиторию.
\end{itemize}

\chapter{Листинг кода}
\lstset{inputencoding=utf8, extendedchars=\true}
\lstinputlisting[language=C++,
                 basicstyle=\ttfamily\scriptsize]{../lab2.py}

\chapter{Пояснение}


\chapter*{Выводы}
\addcontentsline{toc}{chapter}{Выводы}



\end{document}
